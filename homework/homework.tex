\documentclass{scrartcl}
% packages and settings for graphics
\usepackage[pdftex]{graphicx}
\graphicspath{{./}}
\DeclareGraphicsExtensions{.png}
\usepackage[final]{pdfpages}
\usepackage[margin=0.8in]{geometry}

\title{CSE705 Assignment 1 - Homework}
\begin{document}
\setcounter{secnumdepth}{0}
\setlength{\parskip}{10pt plus 1pt minus 1pt}

\section{Question 1}

\subsection{1}
\includegraphics[width=3.6in]{q1model}
\newline
To verify the property $AF(\neg P1 \wedge \neg P2)$ we first need to transform it into the form of $EGf1$:

$\neg EG(\neg ( \neg P1 \wedge \neg P2)) \equiv \neg EG((P1 \vee P2))$.

We then apply the checkEG() algorithm: We identify states that satisfy $(P1 \vee P2)$: $S'$ = $\{0, 2, 3, 4, 5, 6, 7\}$
We form SCCs from these states = \{\{0\}, \{2, 3, 4, 5, 6, 7\}\}.
We pick an SCC and label all its states with $EG((P1 \vee P2))$.
We then perform backwards reachability and find that we can't find any other states in $S'$ that haven't been labeled and are reachable.
We label all other states with $\neg EG((P1 \vee P2))$ and we finish.

The final state of the model after performing checkEG() is:

\includegraphics[width=3.6in]{q1modelchecked}

We can deterimine then that the model does not satisfy the property $AF(\neg P1 \wedge \neg P2)$

\subsection{2}
Mutual exclusion is a safety property, so we can express it using $AGf$: 
$AG(in\_critical\_section(0)  \vee ...\vee in\_critical\_section(n) \vee critical\_section\_empty)$

Eventual entry to a critical section is a liveness property: $AG(AF(in\_critical\_section(0)) \wedge ... \wedge AF(in\_critical\_section(n)))$



\section{Question 2}
\section{Question 3}
\section{Question 4}


\end{document}